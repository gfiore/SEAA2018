
\documentclass[letterpaper, 10 pt, conference]{ieeeconf}
\usepackage{graphicx}
\usepackage{amsmath}
\usepackage{amsfonts}
\usepackage{amssymb}
%\usepackage{graphicx,psfrag,epsfig,epsf}
%\usepackage{verbatim}
%\usepackage{latexsym}
\usepackage{color}
%\usepackage[boxed]{algorithm2e}
\usepackage{diagrams}
%\usepackage{mathabx}
%\usepackage{tikz}
%\usetikzlibrary{automata}
%\usepackage{stackrel}
\usepackage{xcolor}
\usepackage{enumitem}

\newtheorem{example}{Example}
\newtheorem{proposition}{Proposition}
\newtheorem{assumption}{Assumption}
\newtheorem{theorem}{Theorem}
\newtheorem{corollary}{Corollary}
\newtheorem{lemma}{Lemma}
\newtheorem{conjecture}{Conjecture}
\newtheorem{definition}{Definition}
\newtheorem{problem}{Problem}
\newtheorem{remark}{Remark}

\newcommand{\Post}{\operatorname{Post}}
\newcommand{\Id}{\mathrm{Id}}
\newcommand{\diag}{\mathrm{diag}}
\newcommand{\vecc}{\mathrm{vec}}
\newcommand{\SCC}{\mathrm{SCC}}
\newcommand{\Leaves}{\mathrm{Leaves}}
\newcommand{\DAG}{\mathrm{DAG}}
\newcommand{\Scc}{\mathrm{Scc}}
\newcommand{\temp}{\mathrm{temp}}
\newcommand{\card}{\mathrm{card}}
\newcommand{\sech}{\mathrm{sech}}
\newcommand{\mc}{\mathcal}
\newcommand{\mbf}{\mathbf}
\newcommand{\mb}{\mathbb}
\newcommand{\eqbyd}{\triangleq}
\renewcommand{\rm}{\textcolor{red}}
\newcommand{\net}{\mathrm{net}}
\newcommand{\tcomplex}{\mathrm{Tcomplex}}
\newcommand{\scomplex}{\mathrm{Scomplex}}
\newcommand{\length}{\mathrm{length}}
\newcommand{\alt}{\mathrm{alt}}
\newcommand{\emb}{\mathrm{Emb}}
\newcommand{\Tr}{\mathrm{Tr}}
\newcommand{\dom}{\mathrm{Dom}}
\newcommand{\codom}{\mathrm{Codom}}
\newcommand{\nb}{\mathrm{T}}
\newcommand{\Ac}{\mathrm{Ac}}
\newcommand{\CoAc}{\mathrm{Coac}}
\newcommand{\Trim}{\mathrm{Trim}}
\newcommand{\degg}{\mathrm{deg}}
\newcommand{\abs}{\mathrm{a}}
\newcommand{\co}{\mathrm{c}}
\newcommand{\q}{\mathrm{q}}
\newcommand{\pp}{\mathrm{p}}
\newcommand{\Split}{\mathrm{Split}}
\newcommand{\rank}{\mathrm{rank}}

\IEEEoverridecommandlockouts                              % This command is only needed if
% you want to use the \thanks command

\overrideIEEEmargins                                      % Needed to meet printer requirements.

\title{\LARGE \bf
Runtime mission reconfiguration via decentralized and distributed architectures
}

\author{Gabriella Fiore, Davide Di Ruscio, Patrizio Pelliccione
\thanks{G.Fiore, and D. Di Ruscio are with the Department of Information Engineering, Computer Science and Mathematics, Center of Excellence for Research DEWS,
	University of L{'}Aquila, 67100, L{'}Aquila, Italy, email: {\tt\footnotesize gabriella.fiore, davide.diruscio@univaq.it}.}%
\thanks{P. Pelliccione is with the Chalmers University of Technology, University of Gothenburg, Sweden, email: {\tt\footnotesize patrizio.pelliccione@cse.gu.se}.}
}

%{\tt\small albert.author@papercept.net}

\begin{document}
	
	\maketitle
	\thispagestyle{empty}
	\pagestyle{empty}
%	
%	\begin{abstract}
%	FLYAQ \cite{FlyAQ:2015},\cite{FLYAQ:2016} is a framework that enables non-expert users to define high-level missions for a network of Unmanned Aerial Vehicles (UAVs). In this short paper we describe an ongoing work aiming at improving the FLYAQ functionalities, including run-time control of mission execution.
%	To this end, we propose decentralized and distributed control architectures based on formal methods and discrete abstractions of continuous systems to control UAVs operations. We also aim at integrating predictability functionalities, with the objective of allowing efficient mission reconfiguration.
%	%integrate the run-time control of mission execution in the FLYAQ functionalities. 
%%	To this end, we propose to apply control theory techniques not only to the physical system but also to the software architecture. In particular, 
%%	We propose decentralized and distributed control architectures based on formal methods and discrete abstractions of continuous systems to control UAVs operations. 
%%	For what concerns the software architecture, our objective is to derive a mathematical model representing its behavior, in order to enhance its self-adaptiveness by means of control theory techniques.  	
%	\end{abstract}
	
%	\smallskip
	
%	\begin{keywords}
%	Control theory, software engineering, decentralized control, symbolic control, Unmanned Aerial Vehicles (UAVs), FLYAQ	
%	\end{keywords}
%
%

\section{Introduction}\label{sec:Introduction}
There exists a huge range of applications based on Unmanned Aerial Vehicles (UAVs), both in the civilian and in the military domains.
Currently, mission's specification, development, and execution are difficult tasks, thus they are reserved to users having a strong technical expertise.
%
FLYAQ \cite{FlyAQ:2015},\cite{FLYAQ:2016} is a framework that enables non-expert users to define high-level missions for a network of UAVs, such as searching for an object in a certain geographical area, or taking pictures while patrolling an area. Once a high-level mission has been specified, the framework automatically generates a detailed flight plan for each UAV of the network. FLYAQ also allows users to specify no-fly zones, obstacles, etc., thus flight plans are generated according to these specifications, while also avoiding collisions among UAVs of the network. 
%
Currently, the framework only allows off-line mission definition by means of a centralized architecture. That is, based on a high-level user-defined mission, a Ground Control Station (GCS) sends a detailed flight plan to each UAV that accomplish its tasks only based on this information. 
%UAVs inside the network are not equipped with additional communication capabilities, 
%thus they cannot perform coordinated operations by sharing information with other UAVs, they only communicate with the GCS. 
For this reason, dedicated flight plans are conservative, in order to avoid vehicles' trajectories collision or predetermined obstacles.

The aim of our current research is to extend the functionalities of FLYAQ by implementing  decentralized and distributed control architectures\footnote{We follow standard practice and we refer to a decentralized architecture to indicate the case where a global specification is concurrently performed by a collection of local controllers that are not allowed to communicate. In this case, local controllers have to agree in advance on the specification to enforce.
Instead, in a distributed control architecture, local controllers are allowed to communicate (for example, by exchanging information with neighbors).}, 
where the high-level mission is concurrently carried out by the team of UAVs, rather than being executed by the GCS in a centralized manner, thus allowing to define and modify mission execution at run-time.
%\begin{itemize}
%	\item implementing a decentralized control architecture, where the high-level mission is concurrently carried out by the team of UAVs, rather than being executed by the GCS in a centralized manner;
%	\item providing communication capabilities to the UAVs, thus allowing a distributed control architecture to define and modify mission execution at run-time, thanks to the communication between local controllers.
%%	\item enhancing self-adaptiveness of the software architecture by means of a control theoretical approach.
%\end{itemize}
%
%We follow standard practice and we refer to a decentralized architecture to indicate the case where a global specification is concurrently performed by a collection of local controllers that are not allowed to communicate. In this case, local controllers have to agree in advance on the specification to enforce.
%Instead, in a distributed control architecture, local controllers are allowed to communicate (for example, by exchanging information with neighbors).
%
In this paper we provide more details on the former approach. 

Based on the requirements enforced by the considered applications (e.g., disaster prevention and management, homeland security), high-level missions must guarantee safety and efficiency of operation. 
As an example, on the one side, we can consider that predicting the future occurrence of critical situations allows performing actions on the system in order to prevent failures, malfunctioning, or abnormal behaviors. On the other side, predicting the forthcoming availability of functional resources allows reconfiguring the system in a more efficient way.
Our objective is to integrate the FLYAQ functionalities with run-time mission reconfiguration, when forthcoming critical situations or available resources are predicted, rather than when they are detected (that is, when a failure already occurred). In particular, we want to characterize the possibility of predicting in advance the occurrence of critical situations with the objective of proactively reconfiguring the system.
%To enable this reconfiguration, 
%we exploit results derived in \cite{LCSS18} for networks of Finite State Machines (FSMs).
%, as described later in this paper.

%This short paper is organized as follows. In Section \ref{sec:FLYAQ} we provide a description of the FLYAQ framework.
In Section \ref{sec:UAVControl} we describe the decentralized/distributed control architecture for UAVs. 
In Section \ref{sec:Predictability} we describe the predictability property.
%In Section \ref{sec:Implementation} we provide an example of the proposed approach. 
Concluding remarks are offered in Section \ref{sec:Conclusion}.

%\section{Notation and preliminary definitions}\label{sec2}
%The symbols $\mathbb{N}$, $\mathbb{Z}$, $\mathbb{R}$, $\mathbb{R}^{+}$ and $\mathbb{R}_{0}^{+}$ denote the set of nonnegative integer, integer, real, positive real, and nonnegative real numbers, respectively. The symbol $0_n$ denotes the origin in $\mathbb{R}^n$. Given $a,b\in \mathbb{Z}$, we denote $[a;b]=[a,b]\cap\mathbb{Z}$. For a finite set $X$, the symbol $\card(X)$ denotes the cardinality of $X$.
%Given a pair of sets $X$ and $Y$ and a relation $\mathcal{R}\subseteq X\times Y$, the symbol $\mathcal{R}^{-1}$ denotes the inverse relation of $\mathcal{R}$, i.e.
%$\mathcal{R}^{-1}=\{(y,x)\in Y\times X:( x,y)\in \mathcal{R}\}$. Given $X'\subseteq X$ and $Y'\subseteq Y$, we denote $\mathcal{R}(X')=\{y\in Y | \exists x\in X' \text{ s.t. } (x,y)\in \mathcal{R}\}$ and $\mathcal{R}^{-1}(Y')=\{x\in X | \exists y\in Y' \text{ s.t. }  (x,y)\in \mathcal{R}\}$.
%Given a function $f:X\rightarrow Y$ and $X'\subseteq X$ the symbol $f(X')$ denotes the image of $X'$ through $f$, i.e. $f(X')=\{y\in Y | \exists x\in X' \text{ s.t. } y=f(x)\}$.
%Given a set $X$, a set $Y\subseteq X\times X$ is said to be symmetric if $(y,y')\in Y$ implies $(y',y)\in Y$. The minimal symmetric set $Y\subseteq X\times X$ containing a set $Z \subseteq X \times X$ is a symmetric set such that $Z\subseteq Y \subseteq Y'$ for any symmetric set $ Y' \subseteq X \times X$ containing $Z$.
%A continuous function $\gamma:\mathbb{R}_{0}^{+}
%\rightarrow\mathbb{R}_{0}^{+}$, is said to belong to class $\mathcal{K}$ if it
%is strictly increasing and \mbox{$\gamma(0)=0$}; $\gamma$ is said to belong to class
%$\mathcal{K}_{\infty}$ if \mbox{$\gamma\in\mathcal{K}$} and $\gamma(r)\rightarrow
%\infty$ as $r\rightarrow\infty$. 
%Given a vector $x\in\mathbb{R}^{n}$ we denote by $\Vert x\Vert$ the infinity norm of $x$. Given $a\in\mathbb{R}$ and $X\subseteq \mathbb{R}^{n}$, the symbol $aX$ denotes the set $\{y\in\mathbb{R}^{n}| \exists x\in X \text{ s.t. } y=ax\}$. Given $\theta\in\mathbb{R}^+$ and $x\in\mathbb{R}^n$, we define 
%$
%\mathcal{B}_{[-\theta,\theta[}^n(x)=
%\left\{
%y\in\mathbb{R}^{n}| y_i \in [-\theta+x_i,\theta+x_i[, i\in[1;n]
%\right\}
%$, 
%where $x_i$ and $y_i$ denote the $i$--th component of vectors $x$ and $y$, respectively.
%Note that for any $\theta\in\mathbb{R}^+$, the collection of sets $\mathcal{B}^n_{[-\theta,\theta[}(x)$ with $x$ ranging in $2\theta \, \mathbb{Z}^n$ is a partition of $\mathbb{R}^n$. We now define the quantization function.
%Given a positive $n\in\mathbb{N}$ and a quantization parameter $\theta\in\mathbb{R}^+$, the quantizer in $\mathbb{R}^n$ with accuracy $\theta$ is a function
%$
%[\,\cdot \, ]_{\theta}^n:\mathbb{R}^n \rightarrow 2\theta \mathbb{Z}^n
%$,
%associating to any $x\in  \mathbb{R}^{n}$ the unique vector $[x]^n_{\theta} \in 2\theta \mathbb{Z}^{n}$ such that $x\in \mathcal{B}^n_{[-\theta,\theta[}([x]^n_{\theta})$.
%Definition of $[\,\cdot \, ]_{\theta}^n$ naturally extends to sets $X\subseteq \mathbb{R}^{n}$ when $[X]^n_{\theta}$ is interpreted as the image of $X$ through function $[\, \cdot \, ]^n_{\theta}$.
%
%A polyhedron $P\subseteq \mathbb{R}^{n}$ is a set obtained by the intersection of a finite number of (open or closed) half--spaces. A polytope is a bounded polyhedron. Given a set $X$, a function $\mathbf{d}:X\times X\rightarrow \mathbb{R}^{+}_{0}\cup\{\infty\}$ is a quasi--pseudo--metric for $X$ if (i) for any $x\in X$, $\mathbf{d}(x,x)=0$ and (ii) for any $x,y,z\in X$, $\mathbf{d}(x,y)\leq \mathbf{d}(x,z)+\mathbf{d}(z,y)$. 
%If condition (i) is replaced by (i') $\mathbf{d}(x,y)=0$ if and only if $x=y$, then $\mathbf{d}$ is said to be a quasi--metric for $X$. If function $\mathbf{d}$ enjoys properties (i), (ii) and property (iii) for any $x,y\in X$, $\mathbf{d}(x,y)=\mathbf{d}(y,x)$, then $\mathbf{d}$ is said a pseudo--metric for $X$. If function $\mathbf{d}$ enjoys properties (i'), (ii) and (iii), it is said a metric for $X$. When function $\mathbf{d}$ is a (quasi) (pseudo) metric for $X$, the pair $(X,\mathbf{d})$ is said a (quasi) (pseudo) metric space. From \cite{QPM}, given a quasi--pseudo--metric space $(X,\mathbf{d})$, a sequence $\{x_{i}\}_{i\in\mathbb{N}_{0}}$ over $X$ is left (resp. right) $\mathbf{d}$--convergent to $x^{\ast}\in X$, denoted $\stackrel [\leftarrow]{}{\lim} x_{i}=x^{\ast}$ (resp. $\stackrel [\rightarrow]{}{\lim} x_{i}=x^{\ast}$), if for any $\varepsilon\in\mathbb{R}^{+}$ there exists $N\in\mathbb{N}_{0}$ such that $\mathbf{d}(x_{i},x^\ast)\leq \varepsilon$ (resp. $\mathbf{d}(x^\ast,x_{i})\leq \varepsilon$) for any $i \geq N$. Given $X\subseteq \mathbb{R}^{n}$ we denote by $\mathbf{d}_{h}$ the Hausdorff pseudo--metric induced by the infinity norm $\Vert \cdot \Vert$ on $2^{X}$; we recall that for any $X_{1},X_{2}\subseteq X$, \mbox{$\mathbf{d}_{h}(X_{1},X_{2}):=\max\{\vec{\mathbf{d}}_{h}(X_{1},X_{2}),\vec{\mathbf{d}}_{h}(X_{2},X_{1})\}$}, where \mbox{$\vec{\mathbf{d}}_{h}(X_{1},X_{2})=\sup_{x_{1}\in X_{1}}\inf_{x_{2}\in X_{2}} \Vert x_{1}-x_{2}\Vert$} is the Hausdorff quasi--pseudo--metric. 

%The aim of this paper, is to propose an extension of the functionalities of FLYAQ by:
%\begin{itemize}
%	\item integrating a decentralized control architecture, where the high-level mission is concurrently performed by a set of UAV
%	\item integrating a decentralized/distributed control architecture, thus allowing to define and modify mission execution at run-time, by providing communication capabilities to the UAVs;
%	\item enhancing self-adaptiveness of the software architecture by means of a control theoretical approach.
%\end{itemize}
%\begin{itemize}
%	\item decentralizing/distributing the task definition, by means of a decentralized/distributed control architecture;
%	\item integrating the possibility of defining and/or modifying mission execution at run-time;
%	\item providing communication capabilities to the UAVs;
%	\item enhancing self-adaptiveness of the software architecture by means of a control theoretic approach.
%\end{itemize}

%In this paper we provide more details on the former approach. In particular, in Section \ref{sec:UAVControl} we describe the decentralized control architecture for UAVs. 
%In Section \ref{sec:SWControl} we specify how to exploit control theory to the aim of achieving better performance in terms of self-adaptation of the software architecture.

%In Section \ref{sec:Control}, we provide more details regarding the hierarchical control strategy.

%\section{FLYAQ Architecture}\label{sec:FLYAQ}
%%[TBD]
%There exists a huge range of applications based on UAVs, both in the civilian and in the military domains.
%Currently, mission's specification, development, and execution are difficult tasks, thus they are reserved to users having a strong technical expertise.
%%
%The FLYAQ platform, shown in Fig. \ref{fig:flyaq} (see \cite{FlyAQ:2015} and \cite{FLYAQ:2016} for further information), has been designed to allow end-users with expertise neither in ICT nor in aerial vehicles dynamics to specify complex missions (e.g., fire-fighters or rescue workers).
%\begin{figure}
%	\centering
%	\includegraphics[width=0.9\linewidth]{fig/Flyaq}
%	\vspace*{-0.3cm}
%	\caption{FLYAQ platform.}
%	\label{fig:flyaq}
%		\vspace*{-0.3cm}
%\end{figure}
%%
%%From the point of view of end users, a mission in FLYAQ is a set of 
%%movement strategies or tasks (e.g., coverage, search for an object) that a team of UAVs has to perform in predetermined geographical areas, and actions to be done while traversing the interested way-points (e.g., taking pictures or recording a video).
%%%
%%To this end, FLYAQ provides an extensible domain specific language called Monitoring Mission Language (MML), that permits to graphically define civilian missions. MML is made up of three layers (see Fig. \ref{fig:flyaq}): (i) to specify the \textit{mission} through the modeling constructs of the language; (ii) to specify the \textit{context} in which the mission has to take place, such as no-fly zones and obstacles; (iii) to define the geographical \textit{map}  where the mission will be executed. 
%
%A mission in FLYAQ is composed of a series of tasks, that can involve more than a UAV, and that are partially ordered. To manage this situation, UAVs exchange synchronization messages. Presently, any other form of communication (e.g., exchange of data messages) among UAVs is denied.
%%
%
%
%Once the high-level mission has been specified, it is automatically translated in way-points and trajectories
% represented in an intermediate language called Quadrotor Behavior Language (QBL), 
% in terms of actions such as land, take off, hover, read from a sensor, and many others. 
%%The automated generation is based on three main concepts: (i) characteristics of the geographical map; (ii) movement strategies of the UAVs; (iii) actions to be performed while visiting way-points.
%%
%Then, the QBL model is sent to the local controller of each UAV. The UAV behavior can be abstracted as a FSM, where each transition corresponds to a QBL action. Hence, the team of UAVs can be modeled by a network of interacting FSMs.
%%This FSM is organized in three parts: (i) mission entering; (ii) mission tasks execution; (iii) mission leaving.
%
%Presently, the FLYAQ architecture is fully centralized, as represented in Fig. \ref{fig:centralized}. 
%This means that the GCS decomposes the high-level mission into detailed flight plans for each UAV that, in turn, executes its own tasks only based on this information received by the GCS. In particular, once the high-level mission has been graphically specified by the user, the GCS derives the QBL actions,
%% (associated to the corresponding MML tasks), 
% that are sent to each UAV's local controller. Communication among UAVs is limited to synchronization messages.
%%
%\begin{figure}
%	\centering
%	\includegraphics[width=0.9\linewidth]{fig/centralized}
%						\vspace*{-0.3cm}
%	\caption{FLYAQ centralized architecture.}
%	\label{fig:centralized}
%	\vspace*{-0.7cm}
%\end{figure}
%
%In the next sections we describe how to extend the FLYAQ functionalities, first by allowing UAVs to derive their own QBL actions, then by also allowing inter-agent communication among UAVs.



\section{UAV Control Strategy}\label{sec:UAVControl}
%In this Section we first provide an overview on the state of the art for decentralized and distributed control strategies for UAVs. We then describe the solution we are going to propose to extend the functionalities of the FLYAQ framework.
%
%
%%\subsection{Control of team of UAVs: State of the Art}\label{subsec:SoA}
%\subsection{Related works}\label{subsec:SoA}
Typical scenarios considered in the implementation of the FLYAQ framework are: disaster prevention and management; homeland security; protection of critical infrastructures; networking and communications; and environmental protection  \cite{FlyAQ:2015}. 
%
%
%For some applications, rather than using a single powerful UAV, a team of small UAVs equipped with certain communication and computational capabilities to cooperatively perform a specific task, is preferred. The benefits of this approach are manifold: greater flexibility and maneuverability, lower costs, increased coverage in monitoring applications, increased capability of transporting objects, to name a few. For this reason, problems such as formation control \textbf{!!!CIT!!!}, cooperative control \textbf{!!!CIT!!!}, cooperative payload transportation \textbf{!!!CIT!!!} have received considerably attention in the literature related to UAVs.
%
%Missions related to these scenarios can be accomplished by making use of a team of UAVs equipped with communication and computational capabilities, and exploiting cooperation among devices inside the fleet, rather than using a single powerful UAV. The benefits of this approach are manifold: greater flexibility and maneuverability, lower costs, increased coverage in monitoring applications, increased capability of transporting objects, to name a few. For these reasons, problems such as formation control (see e.g \cite{Formation:2007}, \cite{Formation:2008}, \cite{Formation2014}, \cite{Dong:Formation} and references therein), cooperative control (see e.g. \cite{Cooperative:2001}), cooperative payload transportation (see e.g. \cite{mellinger2013cooperative}, \cite{DAndrea}) have received considerable attention in the literature related to UAVs.
%
%As an example, in \cite{Dong:Formation} the authors investigate time varying consensus based formation control problems for UAVs swarms.
%The contribution with respect to other works dealing with formation control for UAVs (with different approaches: leader-follower, behavior-based, or virtual structure) mainly resides on the fact that they consider time-varying formation. In this scenario they are able to provide necessary and sufficient conditions for the UAV swarm system to achieve formation. Moreover, rather than simulating or testing indoor the proposed solution, the authors perform outdoor flight experiments. In this work, a single UAV is modeled by a double integrator. This approximation is made possible by considering that formation control is implemented by means of an inner/outer loop structure. The inner loop is responsible for attitude control, whereas the outer loop is responsible for driving the UAV toward the desired position.
%
%Recently, symbolic control has been recognized to be a promising research trend dealing with the control of multi-agent robotic systems, such as UAVs \cite{Belta:Symbolic}. 
%Indeed, in \cite{Belta:Symbolic} the authors offer an overview of symbolic control methodologies applied to robot motion, where the objective is the automatic generation of control laws from high-level specifications given in a human-like language, to enable (team of) robots to accomplish complex tasks. 
%The main motivation  acknowledged in the paper is to allow non-expert users to specify high-level missions in natural language.
%A common approach when dealing with this kind of problems is to use a multi-layer control architecture. For example, the description in \cite{Belta:Symbolic} is based on a three-layers architecture made up of a specification level (where a high-level specification is translated into a temporal logic formula and the set of all possible discrete solutions to the problem is provided), an execution level (where discrete solutions are refined with respect to robot constraints), and an implementation level (where a hybrid robot control strategy is generated, based on the choice of a reference trajectory). 
%Hybrid control strategies are considered, to take into account the more general scenario where both continuous and discrete dynamics can be modeled and controlled.
%
%As symbolic control is based on the notion of discretization, in \cite{Belta:Symbolic} the authors identify two kinds of discretization: environment-driven and control-driven. The former is established on a partition of the environment and robot motion is controlled through adjacent partitions. Specifications are usually specified in terms of Linear Temporal Logic (LTL), however other specification languages could be used. One of the main drawback of environment-driven discretization (as mentioned in the article) resides on the fact that it is restricted to static and a priori known environments. To overcome this issue, control-driven discretization can be used. Its main feature is that the global control task can be split into simpler sub-tasks or motion primitives to be combined together in a proper manner.
%
%Specifications are usually translated into a Linear Temporal Logic (LTL) formula, however other specification languages could be used.
%The main feature of symbolic control strategies is that they are based on a discrete abstraction of the original system, see e.g. \cite{Dimarogonas:IFAC}, \cite{Dimarogonas:ICRA}. 
%In \cite{Dimarogonas:IFAC} the authors describe two classes of approaches based on discrete abstractions, i.e., methods resulting in deterministic abstractions and methods inducing some determinism. In \cite{Dimarogonas:IFAC}  the authors propose a control strategy for non-deterministic transition systems under LTL specifications. The experimental evaluation of the proposed control architecture is carried out by controlling multiple quadricopters (Iris Plus, 3DRobotics) from an off-board computer. Also in this case, each quadricopter is modeled as a 2D first integrator system (with additive disturbances).
%
%The recent work \cite{Dimarogonas:ICRA} is the first attempt in deriving a distributed control strategy for the motion planning of a team of UAVs under LTL specifications with inter-agent collision avoidance.
%%Formation control, cooperative control, inter-agent collision avoidance have received considerably attention in the literature related to UAVs. 
%The authors motivate this choice by considering that, when dealing with complex and high-level specifications such as safety, surveillance and sequencing, tasks are better explained by means of temporal logic languages. For what concerns UAVs, some solutions have been proposed (see references in \cite{Dimarogonas:ICRA}), however they are centralized or make use of discrete models for the agents (thus without taking into account their continuous dynamics). Hence, the authors propose a solution to overcome these drawbacks. In particular, they derive a discrete abstraction of the static workspace and they model each agent as a single integrator. They propose a continuous controller that allows agents to move between adjacent partitions while avoiding inter-agent collisions, thanks to the use of decentralized navigation functions. On top of that, they define a high-level plan (by means of automata-based formal verification methodologies) that makes use of an abstraction of the agents motion into finite transition systems, satisfying a given LTL formula. The proposed control strategy has been experimentally evaluated on a setup involving two remotely controlled IRIS+ quadrotors from 3DRobotics. The solution proposed in \cite{Dimarogonas:ICRA} is distributed in the sense that each agent is only aware of the workspace partition and the position of its neighbors (i.e., other UAVs within its sensing range).
%
%
%\subsection{Our proposal}\label{subsec:Proposal}
Currently, coordination among UAVs is guaranteed by means of a centralized architecture where a GCS sends a flight plan to each UAV, assuring collision and obstacle avoidance. 
To prevent collisions between different trajectories, a geographical region is statically assigned to each UAV in an exclusive manner.
%, thus bringing to a conservative solution, also with respect to unpredictable or non-nominal behaviors.
This brings to a conservative solution, that is FLYAQ does not allow promptly reacting to undesired situations such as the loss of a UAV, moving obstacles, incomplete or inaccurate context specification, or the availability of new functional resources, and any other occurrence requiring to manage a dynamically varying environment.

%To overcome conservativeness and to provide the possibility of controlling mission execution at run-time, we propose to complement FLYAQ functionalities with a hierarchical control architecture where: (i) the low-level control is essentially an attitude control performed by each UAV in an independent manner (i.e., independent with respect to other UAVs belonging to the fleet); (ii) the high-level control is performed by each UAV or by UAVs having higher priority among others (e.g., equipped with higher computational power) and it allows accomplishing high-level missions (e.g. formation control) in a decentralized/distributed manner.
%\begin{itemize}
%	\item the low-level control is essentially an attitude control performed by each UAV in an independent manner (i.e., independent with respect to other UAVs belonging to the fleet);
%	\item the high-level control is performed by each UAV or by UAVs having higher priority among others (e.g., equipped with higher computational power) and it allows to accomplish high-level missions (e.g. formation control) in a decentralized/distributed manner.
%\end{itemize}
%
%\begin{figure}
%	\centering
%	\includegraphics[width=0.7\linewidth]{fig/Control_architecture}
%	\caption{The proposed hierarchical control architecture. Low Level Control: attitude control running in each UAV. High Level Control: high-level tasks missions performed in a decentralized/distributed way.}
%	\label{fig:controlarchitecture}
%		\vspace*{-0.4cm}
%\end{figure}
%The low-level control locally (on-board) runs in each UAV.
%It is constituted by a local attitude control (which any commercial UAV is usually equipped with) and it is of paramount importance to stabilize the UAV. 
%In this paper we focus our attention on the latter.
%
%The high-level control can be performed by each UAV or by a subset of UAVs, e.g. having greater computational capabilities with respect to others, or having (temporarily or permanently) higher priority. This control is responsible for high-level missions (e.g., formation control). 

The purpose of our current research is to endow FLYAQ with two different control architectures, namely: (i) decentralized, when the UAVs are not able to communicate with each other, but they have to agree in advance on the high-level mission to be carried out; (ii) distributed, when the UAVs are equipped with additional communication capabilities and they can share information, while carrying out a mission.
%\begin{itemize}
%	\item decentralized, when the UAVs are not able to communicate with each other, but they have to agree in advance on the high-level mission to be carried out (see Fig. \ref{fig:decentralized-architecture});
%	\item distributed, when the UAVs are equipped with additional communication capabilities and they can share information, while carrying out a mission (see Fig. \ref{fig:distributed-architecture}).
%\end{itemize}
%
%\begin{figure}
%	\centering
%	\includegraphics[width=0.9\linewidth]{"fig/decentralized architecture"}
%				\vspace*{-0.4cm}
%	\caption{Decentralized control architecture.}
%	\label{fig:decentralized-architecture}
%			\vspace*{-0.3cm}
%\end{figure}
%%
%\begin{figure}
%	\centering
%	\includegraphics[width=0.9\linewidth]{"fig/distributed2"}
%					\vspace*{-0.4cm}
%	\caption{Distributed control architecture.}
%	\label{fig:distributed-architecture}
%			\vspace*{-0.6cm}
%\end{figure}
%
To make a comparison with the centralized architecture, in the decentralized strategy the high-level mission is sent to each UAV that, in turn, derives its own flight plan, based on the off-line knowledge of the environment, obstacles, the number of UAVs in the team, etc. This strategy is particularly useful when UAVs are not equipped with additional communication capabilities and they cannot exchange information while carrying out the mission. 
%The decentralized architecture is especially useful for safety critical applications, when the design choice is to limit as much as possible communication and interaction between UAVs, with the scope of having a fully controlled communication, see \cite{FLYAQ:2016}.
%
On the contrary, when UAVs are allowed to communicate, a distributed control architecture can be implemented. The main advantage with respect to the decentralized case is that UAVs do not need to agree off-line on the high-level mission. The possibility of sharing information among UAVs enable to on-line handle local flight plans, to manage dynamical varying environments, device's failures, and mission's changes. 
%In fact, in this scenario the GCS sends the high-level mission to the UAVs. However, the generation of QBL actions is performed by the UAVs in a fully cooperative manner.
%
Depending on the requirements of the considered scenarios, the advantages of the decentralized and distributed architectures with respect to a centralized one are manifold. Indeed, by allowing UAVs to concurrently carrying out high-level missions, they provide greater flexibility and robustness with respect to external disturbances and failures, and they give the possibility of carrying out a mission with relaxed specifications even when the nominal initial conditions are violated.

We propose to exploit a decentralized symbolic control strategy (see \cite{Pola:TAC2016}, \cite{Pola:Decentralized} for more details), based on formal methods and discrete abstractions of continuous systems. 
The main feature of this approach is that it relies on the discrete abstraction of a continuous system, where an abstract state corresponds to an aggregate of continuous states. Once a discrete abstraction has been derived, well developed methodologies for supervisory control of discrete event systems can be applied.
Also, the choice of applying a symbolic control strategy is motivated by the fact that symbolic methods allow  enforcing complex specifications expressed as regular languages or in linear temporal logic that are difficult to enforce by means of classical control theory techniques \cite{Pola:Decentralized}. 
%As an example, we can mention reachability specifications, control specifications involving a sequence of subtasks that need to be executed in a certain order and can be described by regular languages, or state-based switching specifications to address the situation where a particular control strategy in a set of known ones must be selected depending on the state space's region (see \cite{Tabuada:TAC2008} for more details).
%We exploit results derived in \cite{Pola:TAC2016} and \cite{Pola:Decentralized}, where, respectively, the authors propose networks of symbolic models approximating (for any desired accuracy) networks of discrete-time nonlinear control systems 
%and a decentralized control architecture, in which local controllers contribute concurrently in enforcing the global specification.
%
The main contribution of the strategy we propose is to complement the FLYAQ functionalities with run-time mission reconfiguration to enhance safety and efficiency properties, thus in the next section we introduce the notion of predictability in terms of both critical situations and forthcoming availability of functional resources.


\section{Critical predictability for mission reconfiguration}\label{sec:Predictability}
As teams of UAVs usually operate in safety-critical environments with humans in-the-loop, one of the main challenge is to ensure safety of the overall system, and to increase its robustness with respect to degradations, failures, and any other condition affecting the stability of the operation. 
%Moreover, as team of UAVs usually operate in safety-critical environments with humans in-the-loop, it is fundamental to enforce the mission's compliance with strict timing constraints. 
%
%The most challenging requirement when dealing with UAVs is safety (especially for operations addressed in the FLYAQ framework).
In this scenario predicting the future occurrence of particular states of interest is of paramount importance. 
This subset of states is called critical set and it may represent faulty states, unsafe operations or, more generally, any subset of states which is of particular interest from the system's behavior point of view. We refer to the predictability of both abnormal behaviors (to perform actions on the system in order to prevent critical situations such as failures) and forthcoming availability of functional resources (to reconfigure the system in a more efficient way).
%As an example, on one side, predicting the future occurrence of critical states allows to perform actions on the system in order to prevent critical situations such as failures, malfunctioning, or abnormal behaviors. On the other side, predicting the forthcoming availability of functional resources allows to reconfigure the system in a more efficient way.

Predictability has been studied for both continuous systems and Discrete Event Systems (DES), see e.g. \cite{Laf:2009}, and references therein.
%In the recent paper \cite{Wodes18} the authors introduce also the notions of critical and eventual predictability for Finite State Machines (FSMs), by also providing algorithms to check their validity. 
%Predictability for distributed and decentralized scenarios is investigated in e.g. \cite{Dec:TAC12}, \cite{Dec:Automatica}. These works consider an architecture consisting of a monolithic system and local predictors, each one performing limited observations on the monolithic system, and exchanging information to issue an alarm. 
%However, these results cannot be directly applied to our framework, where a mission is concurrently carried out by multiple UAVs.
%
As pointed out in \cite{YIN2017199}, the problem of verifying the predictability property in a decentralized manner for networks of DES running synchronously has not been addressed, yet. Motivated by this need, in \cite{LCSS18}
 the authors consider the decentralized predictability for networks of DES. 
% The notion of critical predictability is considered, which requires prediction of critical states at every occurrence. 
 A decentralized architecture for critical predictors of the network is proposed, in the sense that prediction is performed by a collection of local predictors, each associated with a DES in the network. 
% It is also shown that predictability of each FSM is not necessary for the whole network to be critically predictable (see Fig. \ref{fig:m1m2} and Fig. \ref{fig:m12}). 
% 	\begin{figure}
% 	\vspace*{0.5cm}
% 	\centering
% 	\includegraphics[width=1\linewidth]{fig/M1M2_png_2}
% 						\vspace*{-0.7cm}
% 	\caption{Example 1. FSMs $M_1$ (left, not critically predictable) and $M_2$ (right, critically predictable).}
% 	\label{fig:m1m2}
% 	\vspace*{-0.2cm}
% \end{figure}
% %
% %	\begin{figure}
% %	\centering
% %	\includegraphics[width=1\linewidth]{fig/M1M2_jpg}
% %	\caption{Example 1. FSMs $M_1$ (left) and $M_2$ (right).}
% %	\label{fig:m1m2}
% %	\end{figure}
% %	
% \begin{figure}
% 	\centering
% 	\includegraphics[width=0.4\linewidth]{fig/M12_png}
% 					\vspace*{-0.3cm}
% 	\caption{Example 1. Parallel composition $M_1 \| M_2$ (critically predictable).}
% 	\label{fig:m12}
% 	\vspace*{-0.6cm}
% \end{figure}	
% This condition is particularly meaningful in our scenario, because this means that the predictability property can be enforced by the cooperation of different UAVs (as an example, consider the case where a single UAV cannot predict a failure of one of its actuators, but other UAVs can predict it and respond accordingly).
%
Furthermore, by applying bisimulation theory, the original network is reduced to a smaller one with the positive effect of lowering complexity. 
%
In particular, equivalence classes on the set of DES are defined, based on the bisimulation relation, and a representative DES is associated with each equivalence class. Finally, the critical predictability property can be tested on the quotient network composed by representatives DES, that is smaller than the original one (when a bisimulation relation can be found between different DES). When the network is critically predictable, \cite{LCSS18} proves that a network of critical predictors (each one associated with an FSM) is equivalent to a centralized predictor.


\section{Conclusions}\label{sec:Conclusion}
The strategy we propose in this extended abstract to improve the FLYAQ functionalities is sound as it is based on: (i) results derived in \cite{Pola:TAC2016} and \cite{Pola:Decentralized} for decentralized control of networks of symbolic models approximating nonlinear systems; and (ii) results derived in \cite{LCSS18} for predictability of networks of FSMs. We point out that the main advantage offered by the predictability property is to promptly and efficiently reconfigure mission control when a critical situation is predicted, both in terms of forthcoming failures of available resources.
%The objective of our future research is to implement decentralized and distributed control architectures for safety-critical applications, where the high-level mission is concurrently carried out by the agents, rather than being executed by a centralized controller.  
Despite the fact that here we focus our attention on team of UAVs, the approach we propose can be used for any multi-agent system (e.g., multi-robots system).
%In \cite{LCSS18} interaction among FSMs is represented by the notion of parallel composition, but the approach can be extended to other topologies of interaction among the agents. Hence, we believe that our approach is flexible enough to tackle with various applications.
%We point out that the main advantage offered by the predictability property is to promptly and efficiently reconfigure mission control when a critical situation is predicted, both in terms of forthcoming failures of available resources.
Indeed, our future research will be devoted to the implementation of the proposed strategy, and to the evaluation of its effectiveness by simulation on real robots.

\bibliographystyle{abbrv}
%\tiny
\bibliography{biblio,biblio2,biblio1,biblio_Wodes}

\end{document}
